\documentclass[12pt]{article}
\usepackage{amsmath}
\usepackage{graphicx}

\renewcommand*{\thesection}{\alph{section}.}

\author{Sundaresan G}
%\title{Advertisement No. CDS/KA/SERB-SRG/DEC2020/RA}
\begin{document}
	
\begin{titlepage}
	\begin{center}
		\vspace*{1cm}
		
		\textbf{\Large{Solution to the 1D Heat Equation}}
		
		\vspace{0.5cm}
		Advertisement No. CDS/KA/SERB-SRG/DEC2020/RA
		
		\vspace{1.5cm}
		
		A report submitted for the application of\\
		Project Assistant Position
		
		
		\vfill
		
		\textbf{\large{Sundaresan G}}\\
		Scientist/Engineer\\
		ISRO Propulsion Complex
		
		\vspace{0.8cm}
		
		B.Tech Mechanical Engineering\\
		Batch of 2016, NITT\\
		
	\end{center}
\end{titlepage}
	\section{Analytical Solution}
	\begin{center}
		\begin{tabular}{l l}
			Given problem & $u_t=u_{xx}$\\		
			Domain & $x=[0,1]$\\
			Boundary condition & $u(0,t)=u(1,t),\
			u_x(0,t)=u_x(1,t)$\\
			Initial condition & $u(x,0)=sin(2\pi x)$
		\end{tabular}
	\end{center}
	The given problem is a well posed homogenous linear PDE and has a unique solution for the given boundary (Neumann) and initial conditions.\\
	The solution is derived as follows by variable separation:\\
	
	\begin{tabular}{r l}
		Let & $u(x,t)=X(x)*T(t)$ \\
		& $u_t=u_{xx} \implies XT_t=X_{xx}T$ \\
		& ${X_{xx}\over X} = {T_t \over T} = \alpha$
		
				
	\end{tabular} \\
	($\alpha$ is a\ constant since LHS is only a function of x and RHS is a function of t)\\
	
	\begin{tabular}{l l}
		Case 1: & $\alpha = 0$ \\
		$\implies$ & $X=ax+b, \ T=c$ \\
		$\implies$ & $u^{\alpha=0}=a_1 x + a_2$
	\end{tabular} \\
	
	\begin{tabular}{l l}
		Case 2: & $\alpha = \beta^2 > 0$ \\
		$\implies$ & $X=ae^{\beta x}+be^{-\beta x}, \ T=ce^{\beta t^2}$ \\
		$\implies$ & $u^{\alpha>0}=(b_1e^{\beta x}+b_2e^{-\beta x})e^{\beta t^2}$
	\end{tabular} \\

	where $b_1$ and $b_2$ are arbitary constants depending on $\beta$ \\
	
	\begin{tabular}{l l}
		Case 3: & $\alpha = -\lambda^2 < 0$ \\
		$\implies$ & $X=acos({\lambda x})+bsin({\lambda x}), \ T=ce^{-\lambda t^2}$ \\
		$\implies$ & $u^{\alpha<0}=(c_1cos({\lambda x})+c_2sin({\lambda x}))e^{-\lambda t^2}$
	\end{tabular} \\
	
	where $c_1$ and $c_2$ are arbitary constants depending on $\lambda$ \\
	
	Due to the homogenous and linear nature of the given PDE, the solution is given by \\ \\ 
	
	\begin{tabular}{r l l}
		$u(x,t)$ &=& $a_1 x + a_2 + \sum_{\beta \in S_1}^{}((b_1(\beta)e^{\beta x}+b_2(\beta)e^{-\beta x})e^{\beta t^2})$\\
		&& + $\sum_{\lambda \in S_2}^{}((c_1(\lambda)cos({\lambda x})+c_2(\lambda)sin({\lambda x}))e^{-\lambda t^2})$
	\end{tabular} \\ \\ 
	where $S_1$ and $S_2$ are countable subsets of Real numbers (since the summation should converge) \\ 
	
	Now the solution is finite even for $t \to \infty$ (by Physics of the heat equation with no external energy source). Hence $b_1$ and $b_2$ should be 0 for any value of $\beta$. \\
	
	$u(x,t) = a_1 x + a_2 + \sum_{\lambda \in S_2}^{}((c_1(\lambda)cos({\lambda x})+c_2(\lambda)sin({\lambda x}))e^{-\lambda t^2})$ \\
	
	Applying condition $u(0,t) = u(1,t)$, we get
	
	$a_2 + \sum_{\lambda \in S_2}^{}((c_1(\lambda))e^{-\lambda t^2}) = a_1 + a_2 + \sum_{\lambda \in S_2}^{}((c_1(\lambda)cos({\lambda })+c_2(\lambda)sin({\lambda}))e^{-\lambda t^2})$
	
	$\implies a_1 + \sum_{\lambda \in S_2}^{}([c_1(\lambda)(cos({\lambda })-1)+c_2(\lambda)sin({\lambda})]e^{-\lambda t^2}) = 0$ \\ 
	
	Since the above equation is true for any value of t, 
	
	\begin{center}
		$a_1 = 0$ \hspace{30mm} and
	\end{center}	
	\begin{equation}
	c_1(\lambda)(cos({\lambda })-1)+c_2(\lambda)sin({\lambda}) = 0 \hspace{20mm}  \text{for every $\lambda \in S_2$}
	\end{equation} 
	
	
	
		
	
	
	
\end{document}